% -*- LaTeX -*-
%
% michael a.g. aïvázis <michael.aivazis@para-sim.com>
% copyright 2024 all rights reserved

% uses the IGARSS-2024 templates
%   spconf.sty  - LaTeX style file
%   IEEEbib.bst - IEEE bibliography style file.

\documentclass{article}
\usepackage{spconf,amsmath,epsfig,url}

% macros
\def\graphql{{\tt graphql}}
\def\pyre{{\tt pyre}}
\def\qed{{\tt qed}}
\def\react{{\tt react}}
\def\relay{{\tt relay}}

% front matter
\title{QED: A NEW OPEN-SOURCE VISUALIZATION TOOL FOR LARGE COMPLEX IMAGE DATASETS
GENERATED BY THE NISAR MISSION}
\name{Michael A.~G.~A\"iv\'azis}
\address{
  ParaSim Inc \\
  Los Angeles, CA \\
  \url{michael.aivazis@para-sim.com}
}

\begin{document}
%
\maketitle
%
\begin{abstract}
Raster datasets come in all shapes and sizes, in a variety of formats, and spanning a
broad range of internal layouts and data types, including complex numbers such as
synthetic aperture radar images. This complexity, confounds many image display systems.
Users are often required to wait for a full image to be processed even to view a small
subset of the image, or to explicitly subset the data in advance. With the rapid growth of
cloud-based archives with multi-gigabyte files, this delay can be exacerbated, and many
display tools simply cannot function when data is not local. We describe a new open-source
visualization tool called \qed that has been designed from the ground up to address these
big data visualization issues. \qed\ is built on a client-server model, and can handle
arbitrarily large data files in a variety of popular formats and data layouts. While
explicitly written to read and display the NASA-ISRO SAR Mission (NISAR) complex data sets
in HDF5 formats, it supports cloud optimized GeoTIFFs, generic flat files that use
standard data types. The server is written in python and C++ using the \pyre\
computational framework. The client user interface runs in a web browser and is written in
javascript using the popular \react\ framework. The communication between the client and
the server utilizes \graphql\ and \relay. The server can be deployed on a cloud instance,
with image data served quickly over the network for smooth scrolling of very large images
without the implicit need for image resolution pyramiding. The code is open source and
extensible to other data formats, types, layouts, and dimensionality. The tool will be
used by the NISAR project and science community for rapid visualizations of the vast data
archive NISAR will generate.
\end{abstract}
%
\begin{keywords}
Image visualization, complex images, visualization frameworks, python frameworks
\end{keywords}
%
\section{Introduction}
\label{sec:intro}
Remote sensing scientists have access to enormous volumes of imagery from international
archives, such as the Alaska Satellite Facility~\cite{ASF} and the ERS~\cite{ERS} data
archives. They have access to sophisticated tools for filtering, reducing, analyzing, and
otherwise manipulating large volumes of data, but the available visualization tools are
generally insufficient for handling complex data. For example, the pixels of Synthetic
Aperture Radar (SAR) images are a complex numbers, with a real and imaginary part. The
magnitude represents the reflectivity of the surface being imaged, and the phase is a
combination of surface scattering phase and the propagation delay from the sensor to the
surface. Both pieces of information are very important for understanding the coherence
properties of the data, but virtually no tools exist to visualize the data. Image data are
delivered in a wide variety of containers -- GeoTIFF, cloud optimized GeoTIFF, HDF5,
flat-binary files with a variety of interleaving schemes and auxiliary metadata files --
and with a variety of data representations. To take full advantage of these files, it is
important to parse and expose the metadata provided in the data file in an intelligent
way, that will allow that

\section{The QED concept}
\label{sec:concept}

\section{The QED interface}
\label{sec:interface}

\section{The development framework}
\label{sec:framework}

\section{Examples}
\label{sec:examples}

\section{Cloud considerations}
\label{sec:cloud}

\section{Future work}
\label{sec:future}

% bibliography
\bibliographystyle{IEEEbib}
\bibliography{refs}

\end{document}
